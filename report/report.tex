\documentclass[twocolumns]{IEEEtran}
\usepackage{graphicx}
\usepackage{amsmath}
\usepackage{authblk}
\usepackage[english]{babel}
\usepackage{blindtext}
\usepackage[ruled,vlined,linesnumbered]{algorithm2e}
\usepackage{algorithmic,float}
\usepackage{setspace}
\usepackage{amsfonts}

\usepackage[utf8]{inputenc}
\title{Parallelization of the Floyd-Warshall algorithm}
\author{David Bertoldi -- 735213 \\ email: d.bertoldi@campus.unimib.it}
\affil{Department of Informatics, Systems and Communication}
\affil{University of Milano-Bicocca}
\date{June 2020}
\begin{document}
\maketitle 
\begin{abstract}
The well known Floyd-Warshall (FW) algorithm solves the all-pairs shortest path problem on directed graphs. In this work we parallelize the FW using three different
programming environments, namely MPI, OpenMP and CUDA. We experimented with multiple data sizes, in order to gain insight on the execution behavior
of the parallelized algorithms on modern multicore and distributed platforms, and on the programmability of the aforementioned environments. We were able
to significantly accelerate FW performance utilizing the full capacity provided by the architectures used.
\end{abstract}
\section{Introduction and Background}
The FW is a classic dynamic programming algorithm that solves the \emph{all-pairs shortest path (APSP)} problem on directed weighted
graphs $G(V, E, w)$, where $V = \{1, \dots, n\}$ is a set of nodes, $E \subseteq V \times V$ are the edges and $w$ is a weight function $E \rightarrow  \mathbb{R}$
that expresses the cost of traversing two nodes. The number of nodes is denoted by $n$ and the number of edges by $m$ . \par
The output of the algorithm is typically in matrix form: the entry in the $i$th row and $j$th column is the weight of the shortest path between
nodes $i$ and $j$. FW runs in $\Theta(|V|^3)$ time and for this reason is a good choiche when working with dense graph: even though there
may be up to $\Omega(|E|^2)$ edges, the computational time is independent from the number of edges. \par
The FW algorithm is shown in Alg. \ref{alg:fw1}

\begin{algorithm}[h!]
\label{alg:fw1}
\SetAlgoLined

\For{$(u, v) \in E$}{
    $M_{u, v} \leftarrow w(u, v)$
}
\For{$v = 1 \rightarrow n$}{
    $M_{v, v} \leftarrow 0$
}
 \For{$k = 1 \rightarrow n$}{
  \For{$i = 1 \rightarrow n$}{
  \For{$j = 1 \rightarrow n$}{
  \If{$M_{i, j} > M_{i, k} + M_{k, j}$}{
 
    $M_{i, j} \leftarrow M_{i, k} + M_{k, j}$ 
 }
 }
 }
 }
 
 \caption[alg:fw1]{The Floyd-Warshall (FW) algorithm}
\end{algorithm}

It is easy to notice that that the nested $i$ and $j$ for-loops are totally independent and therefore parallelizable.


\section{Methodology}



\bibliographystyle{ieeetr}
\bibliography{Bibliography}
\end{document}