\documentclass[compsoc]{IEEEtran}
\usepackage{graphicx}
\usepackage{amsmath}
\usepackage{authblk}
\usepackage[english]{babel}
\usepackage{blindtext}
\usepackage[ruled,vlined,linesnumbered]{algorithm2e}
\usepackage{algorithmic,float}
\usepackage{setspace}
\usepackage{amsfonts}
\usepackage{hyperref}
\graphicspath{ {./images/} }
\usepackage{fontspec}
\usepackage{listings}
\usepackage{amsmath}
\usepackage{mathabx}
\usepackage[bottom]{footmisc}
\newfontfamily\listingsfont[Scale=.7]{inconsolata}\usepackage[font=footnotesize,labelfont=bf]{caption}
\captionsetup[algorithm2e]{font=footnotesize}
\usepackage[table,xcdraw]{xcolor}
\usepackage[utf8]{inputenc}
\title{Parallelization of the Floyd-Warshall algorithm}
\author{David Bertoldi -- 735213 \\ email: d.bertoldi@campus.unimib.it}
\affil{Department of Informatics, Systems and Communication}
\affil{University of Milano-Bicocca}
\date{June 2020}

\definecolor{mGreen}{rgb}{0,0.6,0}
\definecolor{mGray}{rgb}{0.5,0.5,0.5}
\definecolor{mPurple}{rgb}{0.58,0,0.82}
\definecolor{backgroundColour}{rgb}{0.95,0.95,0.92}

\lstdefinestyle{CStyle}{
    backgroundcolor=\color{backgroundColour},   
    commentstyle=\color{mGreen},
    keywordstyle=\color{magenta},
    keywordstyle=[2]{\color{lime}},
    morekeywords=[2]{;},
    numberstyle=\tiny\color{mGray},
    stringstyle=\color{mPurple},
    basicstyle=\linespread{1}\listingsfont,
    breakatwhitespace=false,         
    breaklines=true,                 
    captionpos=b,                    
    keepspaces=true,                 
    numbers=none,                    
    numbersep=5pt,                  
    showspaces=false,                
    showstringspaces=false,
    showtabs=false,                  
    tabsize=2,
    language=C++
}

\begin{document}

\maketitle 



\begin{abstract}
The well known Floyd-Warshall (FW) algorithm solves the all-pairs shortest path problem on directed graphs. In this work we parallelize the FW using three different
programming environments, namely MPI, OpenMP and CUDA. We experimented with multiple data sizes, in order to gain insight on the execution behavior
of the parallelized algorithms on modern multicore and distributed platforms, and on the programmability of the aforementioned environments. We were able
to significantly accelerate FW performance utilizing the full capacity provided by the architectures used.
\end{abstract}



\section{Introduction and Background}
The FW is a classic dynamic programming algorithm that solves the \emph{all-pairs shortest path (APSP)} problem on directed weighted
graphs $G(V, E, w)$, where $V = \{1, \dots, n\}$ is a set of nodes, $E \subseteq V \times V$ are the edges and $w$ is a weight function $E \rightarrow  \mathbb{R}$
that expresses the cost of crossing two nodes. The number of nodes is denoted by $n$ and the number of edges by $m$ . \par
The output of the algorithm is typically in matrix form: the entry in the $i$th row and $j$th column is the weight of the shortest path between
nodes $i$ and $j$. FW runs in $\Theta(|V|^3)$ time and for this reason is a good choiche when working with dense graph: even though there
may be up to $\Omega(|E|^2)$ edges, the computational time is independent from the number of edges. \par
The FW algorithm is shown in \textbf{Algorithm \ref*{alg:fw1}}.

\begin{algorithm}[h!]

\SetAlgoLined

\For{$(u, v) \in E$}{
    $M_{u, v} \leftarrow w(u, v)$
}
\For{$v = 1 \rightarrow n$}{
    $M_{v, v} \leftarrow 0$
}
 \For{$k = 1 \rightarrow n$}{
  \For{$i = 1 \rightarrow n$}{
  \For{$j = 1 \rightarrow n$}{
  \If{$M_{i, j} > M_{i, k} + M_{k, j}$}{
 
    $M_{i, j} \leftarrow M_{i, k} + M_{k, j}$ 
 }
 }
 }
 }
 
\caption{The Floyd-Warshall (FW) algorithm}\label{alg:fw1}
\end{algorithm}

A C implementation of this algorithm can be found \href{https://github.com/firaja/Parallel-FloydWarshall/blob/master/sequential.c}{here};
this version is referred in this document as \emph{sequential} implementation and it is used as base version when comparing to parallel implementations. \par \par



In particular we define the \emph{speedup} ($S$) of a given version $v$ of the algorithm as it follows:

\[S = \frac{T_{s}}{T_{v}}\]

where $T_v$ is the execution time of the version $v$ of FW and $T_s$ is the execution time of the \emph{sequential} version. \par
Another important factor is the \emph{efficiency} ($E$) which describes how the solution scales well by adding processor units

\[E = \frac{S}{p}\]
where $S$ is the speedup and $p$ the number of processor units; the closer $E$ remains to $1$ while increasing $p$, the better.

\section{Terminology}
In this document we use specific terms that refer to specific definitions. Such terms are \emph{emphasized} in order to distingiush them from everyday terms
and we listed them in this section, which acts as a dictionary.

\subsubsection*{Edge/cell under analysis} 
The edge/cell in the distance matrix selected by FW for a given $i$ and $j$. Often noted as $M_{i,j}$ within this document.

\subsubsection*{Intermediate edges} 
The two edges which sum must be compared with the edge \emph{under analysis} (line 10 of \textbf{Algorithm \ref*{alg:fw1}}).
They are noted as $M_{i,k}$ and $M_{k,j}$ within this document.

\subsubsection*{Serial FW} 
The implementation of FW as show in \textbf{Algorithm \ref*{alg:fw1}}: the execution is performed on a single process/thread.

\subsubsection*{Speedup}
For a given implementation $v$ of FW, it is defined as follows:
\[S_{v} = \frac{T_{s}}{T_{v}}\]
where $T_v$ is the execution time of the version $v$ of FW and $T_s$ is the execution time of the \emph{serial FW}. 

\subsubsection*{Efficiency}
For a given implementation $v$, it is defined as follows:
\[E_v = \frac{S_v}{p}\]
where $S_v$ is the speedup of version $v$ of FW and $p$ is the number of processor units involved in the computation; the closer $E_v$ remains to $1$ while increasing $p$, the better.

\section{Methodology}
It is easy to notice that that the nested $i$ and $j$ for-loops in Alg. \ref*{alg:fw1} are totally independent and therefore parallelizable. \\
In this section we describe three strategies to overcome the iniefficiency due to a mono-thread computation.


\subsection{Distributed with MPI}

....



\begin{lstlisting}[style=CStyle]
...
int processes, rank;

MPI_Init(&argc, &argv);

MPI_Comm_size(MPI_COMM_WORLD, &processes);
MPI_Comm_rank(MPI_COMM_WORLD, &rank);

MPI_Bcast(&n, 1, MPI_INT, 0, MPI_COMM_WORLD);

populateMatrix(matrix, n, density, rank, processes);

int k, i, j, temp;
int* kRow = malloc(n*sizeof(int));
int section = n / processes;

for (k = 0; k < n; k++) 
{
	
	castKRow(matrix, n, section, kRow, k, rank);
	
	for (i = 0; i < section; i++)
	{
		for (j = 0; j < n; j++) 
		{
			temp = matrix[i * n + k] + kRow[j];
			if (temp < matrix[i * n + j])
			{
				matrix[i * n + j] = temp;
			}
		}
	}
}

gatherResult(matrix, n, rank, processes);
\end{lstlisting}





































































\subsection{Multithreading with OpenMP}
OpenMP (Open Multi-Processing) is an application programming
interface (API) for parallel programming intended to work on shared-
memory architectures. More specifically, it is a set of compiler
directives, library routines and environmental variables, which in-
fluence run-time behavior. OpenMP enables parallel programming in
various languages, such as C, C++ and FORTRAN and runs on most
operating systems. \par
The OpenMP API uses the fork-join model of parallel execution.
Multiple threads perform tasks defined implicitly or explicitly by
OpenMP directives. All OpenMP applications begin as a single thread
of execution, called the initial thread. The initial thread executes
sequentially until it encounters a parallel construct. At that point,
this thread creates a group of itself and zero or more additional
threads and becomes the master thread of the new group. Each thread
executes the commands included in the parallel region, and their
execution may be differentiated, according to additional directives
provided by the programmer. At the end of the parallel region, all
threads are synchronized. \par
The runtime environment is responsible for effectively scheduling
threads. Each thread, receives a unique id, which differentiates it
during execution. Scheduling is performed according to memory
usage, machine load and other factors and may be adjusted by altering
environmental variables. In terms of memory usage, most variables in
OpenMP code are visible to all threads by default. However, OpenMP
provides a variety of options for data management, such as a thread-
private memory and private variables, as well as multiple ways of
passing values between sequential and parallel regions. Additionally,
recent OpenMP implementations introduced the concept of tasks,
as a solution for parallelizing applications that produce dynamic
workloads. Thus, OpenMP is enriched with a flexible model for
irregular parallelism, providing parallel while loops and recursive
data structures. \par
The main advantage on using OpenMP is the ease of developing parallelisms
with simple constructs that do not differ too much from the original implementation.

The following snippet shows the implementation used in this work: the matrix containing the
distances between vertices is shared among all the threads, while the 3 nested for-loops are executed
by each thread indipendentely.



\begin{lstlisting}[style=CStyle]
int i, j, k;
#pragma omp parallel num_threads(t) shared(M) private(k) {
	for(k = 0; k < n; k++) {
		#pragma omp for private(i,j) schedule(dynamic)			
		for(i = 0; i < n; i++) {
			for(j = 0; j < n; j++) {
				if(M[i][j] > M[i][k] + M[k][j] || M[i][j] == 0) {
					M[i][j] = M[i][k] + M[k][j];
				}
			}
		}
	}
}
\end{lstlisting}

A dynamic scheduler works better than a static one in this case because the
density of the edges may vary and the compiler cannot foresee the content of the matrix. \\
Also note that the solution does not implement a \texttt{collapse} directive like this:
\begin{lstlisting}[style=CStyle]
...
#pragma omp for ... collapse(3)
for(k = 0; k < n; k++) {			
	for(i = 0; i < n; i++) {
		for(j = 0; j < n; j++) {
			...
		}
	}
}
\end{lstlisting}
that's because when we collapse a multiple loops, OpenMP turns the into a single loop: there is a single
index that is constructed from \texttt{i}, \texttt{i} and \texttt{k} using division and modulo operations. This can
have huge impact on the performance because of the overhead, especially if the matrix is really wide.

\textbf{Figure \ref*{fig:threads}} shows how 2 threads interact inside the matrix: the red and orange zones highlight the cells where
Thread 1 and Thread 2 can write respectively; the blue and turquise cells represent the intermediate vertices that are compared
with the vertice under analysis for Thread 1 and Thread 2 respectively.

\begin{figure}[h!]
\centering                                                                        
\includegraphics[width=3.5in]{diagrams/openmp-threads}
\captionsetup{justification=centering,margin=2cm}                                                                                                                                   
\caption{View of the data each thread can reach}                                                                                                                                            
\label{fig:threads}                                                                                                                                                           
\end{figure}

It may seems that this implementation is affected by data races, because there's no lock when writing in \texttt{M} 
and one of the intermediate cells could be the same under analysis on another thread; for example, Thread 1 could write
in cell $M_{i,j}$ before or after that Thread 2 calculates $M_{x,j} > M_{x,i} + M_{i,j}$, making the result unpredictable.

We prove that no race condition may appear in this case. Let's assume we have 2 threads, namely $T^1$ and $T^2$ and they are
analyzing cell $(i,j)$ and $(x,y)$, with $x \neq i$. We denote the previous statement simply with $T^{1}_{i,j}$ and $T^{2}_{x,y}$.
Having in common $k$, the two threads are now calculating the following system

\begin{flalign}\label{eq:sys1}
 &&  \left\{\begin{matrix}
T^{1}_{i,j} & > & T^{1}_{i,k} & + & T^{1}_{k,j} \\
\\ 
T^{2}_{x,y} & > & T^{2}_{x,k} & + & T^{2}_{k,y}
\end{matrix}\right. &&
\end{flalign}

The two inequalities are taken from the condition of the \emph{if} condition in \textbf{Algorithm \ref*{alg:fw1}}. \\ 
If both are true, then $T^{1}$ and $T^{2}$ are allowed to write in $M$. In order to have a data race condition, the following
must be true
\[(k = i \wedge y = j) \vee (x = i \wedge k=j)\]
but since $x \neq i$, only the following must be verified

\begin{flalign}\label{eq:cond1}
 &&  k = i \wedge y = j &&
\end{flalign}
By applying (\ref*{eq:cond1}) to (\ref*{eq:sys1}) we have

\begin{flalign}\label{eq:sys2}
 &&  T^{1}_{i,j} > T^{1}_{k,k} + T^{1}_{i,j} &&
\end{flalign}
but (\ref*{eq:sys2}) is clearly false, because $M$ is a hollow matrix \emph{i.e.} the  diagonal elements are all equal to $0$, leaving the 
following inequality:
\begin{flalign}\label{eq:sys3}
 &&  T^{1}_{i,j} > T^{1}_{i,j} &&
\end{flalign}
Clearly no number can be greater than itself and this means that only $T^2$ may write in $M$ at this point. \\
No race condition can appear if $k$ is the same among the threads and threrfore there's no need to verify atomicity of the write operation;
the lack of OpenMP directive like \texttt{atomic} or \texttt{critical} plays in favor of performance. \par

The speedup of the solution is sligthy worse than the ideal speedup (see \textbf{Figure \ref*{fig:omp-speedup}})

\begin{figure}[h!]
\centering                                                                        
\includegraphics[width=3.5in]{diagrams/openmp-speedup}
\captionsetup{justification=centering,margin=2cm}                                                                                                                                   
\caption{Speedup of \emph{OpenMP} FW on a octacore CPU}                                                                                                                                            
\label{fig:omp-speedup}                                                                                                                                                           
\end{figure}
When scaling from 7 to 8 threads, we notice a slight deviation from the previous (almost) linear trend. That's because the measurement
is taken from a 8-core/8-thread CPU, namely Intel Core i7-9700K, and because no other cores were free to manage the OS and its subprocesses, the scheduler
divided this task among all the threads. So we have approximated the speedup without counting the fluctuations due to the management of the OS. \par

The efficiency, which stays always above $90\%$, is shown in \textbf{Figure \ref*{fig:omp-efficiency}} alongside with its theoretical counterpart.

\begin{figure}[h!]
\centering                                                                        
\includegraphics[width=3.5in]{diagrams/openmp-efficiency}
\captionsetup{justification=centering,margin=2cm}                                                                                                                                   
\caption{Efficiency of \emph{OpenMP} FW on a octacore CPU}                                                                                                                                            
\label{fig:omp-efficiency}                                                                                                                                                           
\end{figure}
An overview of the timings collected can be found in \textbf{Table \ref*{tab:omp-time}}.













































































\subsection{GPGPU with CUDA}

\section{Computational Platforms and Software Libraries}
This section describes technical details about the implementation of the solutions exposed in this document. It also contains
the collected data used for the analysis.
\subsection{Sequential implementation}
The \emph{sequential} version of the FW algorithm can be found \href{https://github.com/firaja/Parallel-FloydWarshall/blob/master/sequential.c}{here}. 
The program is compiled as follows:
\begin{lstlisting}[basicstyle=\footnotesize\ttfamily]
$ gcc sequential.c -o sequential.out -O3
\end{lstlisting}

notice the \texttt{-O3} flag that makes the program run $3.5$ times faster.
The program accepts 2 arguments:
\begin{lstlisting}[basicstyle=\footnotesize\ttfamily]
$ ./sequential.out <v> <d>
\end{lstlisting}
where \texttt{v} is the number of verteces, expressed ad positive integer, and \texttt{d} is the density of the presence of edges, expressed as an integer from 0 to 100.
\par
The \texttt{gcc} version used for this work is 7.5.0 and the program ran on a Intel Core i7-9700K. \\
Table \ref*{tab:seq-time} shows the execution time (expressed in milliseconds) depending on the number of vertices.


\begin{table}[h!]
\centering
\begin{tabular}{|r|r|}
\hline
\rowcolor[HTML]{3166FF} 
{\color[HTML]{FFFFFF} \textbf{Vertices}} & {\color[HTML]{FFFFFF} \textbf{Execution time}} \\ \hline
1000                                     & 1095 ms                                        \\ \hline
2000                                     & 8860 ms                                        \\ \hline
5000                                     & 138643 ms                                      \\ \hline
7500                                     & 468750 ms                                      \\ \hline
10000                                    & 1112111 ms                                     \\ \hline
12500                                    & 2170138 ms                                     \\ \hline
\end{tabular}
\caption{Execution time of the \emph{sequential} FW}                                                                                                                                            
\label{tab:seq-time} 
\end{table}

Figure \ref*{fig:seq-time} shows the trend of the execution time. 

\begin{figure}[h!]
\centering                                                                        
\includegraphics[width=3in]{images/seq-time}
\captionsetup{justification=centering,margin=2cm}                                                                                                                                   
\caption{Trend of the execution time based on Table \ref*{tab:seq-time}}                                                                                                                                            
\label{fig:seq-time}                                                                                                                                                           
\end{figure}


It is easy to notice that the graph represents a 
third grade curve; this is the interpolated function starting from the collected data:
\[f(n) = 2.22n^3 - 18.83n^2 + 92.14n -94.84 \approx \Theta(n^3) \]



\subsection{MPI implementation}

The \emph{MPI} version of the FW algorithm can be found \href{https://github.com/firaja/Parallel-FloydWarshall/blob/master/mpi.c}{here}. 
The program is compiled as it follows:

\begin{lstlisting}[basicstyle=\footnotesize\ttfamily]
$ mpicc -g -Wall mpi.c -o mpi.out -O3
\end{lstlisting}
Like the \emph{sequential} version, the program accepts 2 arguments:
\begin{lstlisting}[basicstyle=\footnotesize\ttfamily]
$ mpirun -np <p> mpi.out <v> <d>
\end{lstlisting}
The \texttt{MPI} version used for this work is 2.1.1 and the program ran on a cluster of 8 nodes over a LAN. \\
Table \ref*{tab:mpi-time} shows the execution time (expressed in milliseconds) and the percentage of time spent in initialization and communication, depending on the number of processors; 
the program computes the \emph{APSP} problem for 5040 vertices.

\begin{table}[h!]
\centering
\begin{tabular}{|r|r|r|}
\hline
\rowcolor[HTML]{F56B00} 
{\color[HTML]{FFFFFF} \textbf{Processors}} & {\color[HTML]{FFFFFF} \textbf{Execution time}} & {\color[HTML]{FFFFFF} \textbf{MPI \%}} \\ \hline
1                                          & 90842 ms                                               & 0.06                                 \\ \hline
2                                          & 46608 ms                                               & 2.57                                 \\ \hline
3                                          & 31309 ms                                               & 4.45                                \\ \hline
4                                          & 24457 ms                                               & 5.47                                \\ \hline
5                                          & 20200 ms                                               & 6.95                                 \\ \hline
6                                          & 17403 ms                                               & 8.35                                \\ \hline
7                                          & 15563 ms                                               & 9.12                                 \\ \hline
8                                          & 13812 ms                                               & 10.1                                \\ \hline
\end{tabular}
\caption{Execution time of the \emph{MPI} FW}                                                                                                                                            
\label{tab:mpi-time}
\end{table}
Timings are captured through \texttt{mpiP} that calculates the percentage of time spent by MPI for initialization/finalization and communication.
\texttt{mpiP} is a lightweight profiling library for MPI applications. Because it only collects statistical information about MPI functions, \texttt{mpiP} generates considerably less overhead and much less data than tracing tools. All the information captured by mpiP is task-local. It only uses communication during report generation, typically at the end of the experiment, to merge results from all of the tasks into one output file.


\subsection{OpenMP implementation}
The \emph{MPI} version of the FW algorithm can be found \href{https://github.com/firaja/Parallel-FloydWarshall/blob/master/openmp.c}{here}. 
The program is compiled as it follows:
\begin{lstlisting}[basicstyle=\footnotesize\ttfamily]
$ g++ -fopenmp openmp.c -o openmp.out -O3
\end{lstlisting}
Unlike the \emph{sequential} version, the program accepts 3 arguments:
\begin{lstlisting}[basicstyle=\footnotesize\ttfamily]
$ ./openmp.out <v> <d> <t>
\end{lstlisting}
where \texttt{t} is the number of threads OpenMP can use for parallelization. \\
The \texttt{gcc} version used for this work is 7.5.0 and the program ran on a Intel Core i7-9700K, which has 8 core with no Hyper-Threading.
Table \ref*{tab:omp-time} shows the execution time (expressed in milliseconds) depending on the number of vertices and available cores.

\begin{table}[h!]
\centering
\begin{tabular}{|r|r|r|r|}
\hline
\rowcolor[HTML]{CB0000} 
\multicolumn{1}{|c|}{\cellcolor[HTML]{CB0000}{\color[HTML]{FFFFFF} \textbf{Vertices}}} & \multicolumn{1}{c|}{\cellcolor[HTML]{CB0000}{\color[HTML]{FFFFFF} \textbf{2 threads}}} & \multicolumn{1}{c|}{\cellcolor[HTML]{CB0000}{\color[HTML]{FFFFFF} \textbf{4 threads}}} & \multicolumn{1}{c|}{\cellcolor[HTML]{CB0000}{\color[HTML]{FFFFFF} \textbf{8 threads}}} \\ \hline
1000                                                                                   & 541 ms                                                                                         & 364 ms                                                                                         & 158 ms                                                                                          \\ \hline
2000                                                                                   & 4365 ms                                                                                        & 2921 ms                                                                                        & 1260 ms                                                                                         \\ \hline
5000                                                                                   & 69590 ms                                                                                       & 46443 ms                                                                                       & 19496 ms                                                                                        \\ \hline
7500                                                                                   & 230260 ms                                                                                      & 155531 ms                                                                                      & 65118 ms                                                                                        \\ \hline
10000                                                                                  & 543900 ms                                                                                      & 367434 ms                                                                                       & 153682 ms                                                                                       \\ \hline
12500                                                                                  & 1063129 ms                                                                                     & 716363 ms                                                                                       & 299006 ms                                                                                       \\ \hline
\end{tabular}
\caption{Execution time of the \emph{OpenMP} FW}                                                                                                                                            
\label{tab:omp-time}
\end{table}

\subsection{CUDA implementation}
The \emph{MPI} version of the FW algorithm can be found \href{https://github.com/firaja/Parallel-FloydWarshall/blob/master/cuda.c}{here}. 
The program is compiled as it follows:
\begin{lstlisting}[basicstyle=\footnotesize\ttfamily]
$ nvcc cuda.cu -o cuda.out \
  -gencode=arch=compute_75,code=compute_75 -O3
\end{lstlisting}
Unlike the \emph{sequential} version, the program accepts 3 arguments:
\begin{lstlisting}[basicstyle=\footnotesize\ttfamily]
$ ./cuda.out <v> <d> <b>
\end{lstlisting}
where \texttt{b} is the number of threads per block. \\
The \texttt{nvcc} version used for this work is 10.2 and the program ran on a CPU Intel Core i7-9700K and a GPU NVIDIA GeForce RTX 2070 Super.
Table \ref*{tab:cuda-time} shows the execution time (expressed in milliseconds) depending on the number of vertices and the block size of threads.

\begin{table}[h!]
\centering
\begin{tabular}{|r|r|r|r|}
\hline
\rowcolor[HTML]{009901} 
\multicolumn{1}{|l|}{\cellcolor[HTML]{009901}{\color[HTML]{FFFFFF} \textbf{verteces}}} & \multicolumn{1}{l|}{\cellcolor[HTML]{009901}{\color[HTML]{FFFFFF} \textbf{1024 block}}} & {\color[HTML]{FFFFFF} \textbf{256 block}} & \multicolumn{1}{l|}{\cellcolor[HTML]{009901}{\color[HTML]{FFFFFF} \textbf{32 block}}} \\ \hline
1000                                                                                   & 24 ms                                                                                                & 19 ms                                                  & 53 ms                                                                                              \\ \hline
2000                                                                                   & 219 ms                                                                                               & 186 ms                                                 & 356 ms                                                                                            \\ \hline
5000                                                                                   & 2818 ms                                                                                              & 2709 ms                                                & 4966 ms                                                                                            \\ \hline
7500                                                                                   & 9894 ms                                                                                              & 9044 ms                                                & 16297 ms                                                                                           \\ \hline
10000                                                                                  & 22190 ms                                                                                             & 21392 ms                                               & 38208 ms                                                                                           \\ \hline
125000                                                                                 & 42955 ms                                                                                             & 41460 ms                                               & 75034 ms                                                                                           \\ \hline
\end{tabular}
\caption{Execution time of the \emph{CUDA} FW}                                                                                                                                            
\label{tab:cuda-time}
\end{table}


\section*{Conclusion}

In this work we described three approaches to parallelize the FW algorithm with three different architectures:
distributed with MPI, shared-memory multiprocessing with OpenMP and GPGPU with CUDA.

The fastest "pure" implementation is \emph{CUDA FW} thanks to the high computation capability and high memory bandwidth, 
but it requires more expensive hardware. 

\emph{MPI FW} (with one thread per node)  is still faster than the \emph{serial}, but the implementation
is more complex, the cost of the cluster (hosting and maintenance) makes the solution non convenient,
the efficiency depends on the network bandwidth and the speedup is not that high.

\emph{OpenMP FW} is almost 95 times faster than \emph{serial FW} but 13 times slower than \emph{CUDA FW}. The absence of
overhead due to communication, fast memory access, easy development process and relatively low costs make this solution the
most affordable, maintenable and cost-efficient one.

Talking about hybrid solutions, \emph{MPI + CUDA} is obviously faster than \emph{MPI + OpenMP} as long as the matrix is not small, 
but the benefits may not justify the total cost of the infrastructure: the monthly cost for hosting a server with a GPU is at least
four times the cost of one without a GPU. This solution is suggested to those systems that really need the lowest response time possible
(\emph{e.g.} real-time systems or systems that cannot rely on a caching mechanism in front of them).




\bibliographystyle{ieeetr}
\bibliography{Bibliography}

\begin{thebibliography}{9}

\bibitem{nvidia} 
NVIDIA Developer: Turing tuning guide,
\\\texttt{\href{https://docs.nvidia.com/cuda/turing-tuning-guide/index.html}{https://docs.nvidia.com/cuda/turing-tuning-guide/index.html}}


\bibitem{fletcher} 
Fletcher, J. G. (January 1982). \textit{An Arithmetic Checksum for Serial Transmissions}. IEEE Transactions on Communications. COM-30 (1): 247–252. 
\href{https://ieeexplore.ieee.org/document/1095369}{doi:10.1109/tcom.1982.1095369}
\end{thebibliography}


\end{document}









